% This file is part of the Digital Assyriologist.  Copyright
% (c) Steve Tinney, 1994, 1995.  It is distributed under the
% Gnu General Public License as specified in /da/doc/COPYING.
%
% $Id: polyglot.tex,v 0.3 1995/11/12 03:16:13 b Exp $

% This file implements multiple output language and hyphenation patterns
% for the DA TeX format.
%
% There are two issues here. Firstly, control sequences such as @obverse
% produce an actual heading in the output. We need to be able to allow users
% to specify that their document is written in French or whatever and have
% the words produced be appropriate to that language. Secondly, hyphenation
% must be modifiable both at the default level, so that a paper written in
% German will be hyphenated correctly, and on an ad hoc basis so that, for
% example, bibliography entries can be correctly hyphenated be they in 
% English, French, German, Italian or Spanish.
%

% The strategy for language specific label selection is as follows:
%
%	We define a means of specifying a control sequence and its values
%	  in each of the target languages
%	We define a means of selecting a target language as the default
%	Upon selection of the language, we read two files of polyglot 
%	  definitions, the base set in polybase.tex and the user's additions
%	  in polyuser.tex
%
\chardef\English=0
\newlanguage\French
\newlanguage\German
\newlanguage\Italian
\newlanguage\Spanish

% This method means we have only 3 spaces for expansion, but we'll drive off
% that bridge when we come to it.
%
%   #1 = control sequence, without escape character
%   #2 = English language label
%   #3 = French language label
%   #4 = German language label
%   #5 = Italian language label
%   #6 = Spanish language label
%
% We actually define a control sequence \#1str, so that one can define the
% control sequences themselves to have other formatting information, and use
% the string value only where needed, e.g. \def\obverse{\heading{\obversestr}}.
% Of course, there need be no control sequence corresponding to obverse. For
% example, the abbreviations used in running text with @obverse and friends
% are implemented by saying 
%	\polyglot{obverseabb}{obv.}{...} etc
\def\polyglot#1#2#3#4#5#6{
  \ifcase\language
     \expandafter\def\csname#1str\endcsname{#2}% English
  \or\expandafter\def\csname#1str\endcsname{#3}% French
  \or\expandafter\def\csname#1str\endcsname{#4}% German
  \or\expandafter\def\csname#1str\endcsname{#5}% Italian
  \or\expandafter\def\csname#1str\endcsname{#6}% Spanish
  \fi}
\def\polyread{
  \checkfile{polybase}%
  \ifitexists\def\next{\macrofile{polybase}}\else\let\next\relax\fi
  \next
  \checkfile{polyuser}%
  \ifitexists\def\next{\macrofile{polybase}}\else\let\next\relax\fi
  \next }
\def\EnglishLanguage{\language\English \polyread}
\def\FrenchLanguage{\language\French \polyread}
\def\GermanLanguage{\language\German \polyread}
\def\ItalianLanguage{\language\Italian \polyread}
%\def\SpanishLanguage{\language\Spanish \polyread}
\def\EnglishHyphens{\language\English }
%\def\FrenchHyphens{\language\French }
%\def\GermanHyphens{\language\German }
%\def\ItalianHyphens{\language\Italian }
%\def\SpanishHyphens{\language\Spanish }

% Hyphenation, miscellaneous macros, and initial values for standard layout
\EnglishHyphens
\message{English hyphenation}
\lefthyphenmin=2 \righthyphenmin=3 % disallow x- or -xx breaks
%\input hyphen

%\ifISOfonts
%\else
%\FrenchHyphens
%\input hyphfrda
%\fi

%\ifISOfonts
%\else
%\GermanHyphens
%\input hyphgeda
%\fi

%\language\English

%\EnglishLanguage

\def\showhyphens#1{\setbox0\vbox{\parfillskip\z@skip\hsize\maxdimen\tenrm
  \pretolerance\m@ne\tolerance\m@ne\hbadness0\showboxdepth0\ #1}}
\def\nohyphenation{\hyphenpenalty\@M \exhyphenpenalty\@M \finalhyphendemerits\@M}
\def\dohyphenation{\hyphenpenalty50  \exhyphenpenalty50  \finalhyphendemerits5000 }
\endinput

%%%%%%%%%%%%%%%%%%%%%%%%%%%%%%%%%%%%%%%%%%%%%%%%%%%%%%%%%

$Log: polyglot.tex,v $
Revision 0.3  1995/11/12 03:16:13  b
don't include hyphens when using ISO fonts: the hyphenation needs
to use different inputs for different charsets

Revision 0.2  1995/11/09 12:42:15  s
initial checkin

% Revision 0.2  1995/11/09  12:42:15  s
% fixed typo in comment to \polyglot
%
% Revision 0.1  1994/12/12  03:56:05  s
% initial checkin
%
