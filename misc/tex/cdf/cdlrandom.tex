% RANDOM.TEX       v.1   (Donald Arseneau)
% Generating "random" numbers in TeX. 
%
% Random integers are generated in the range 1 to 2147483646 by the
% macro \nextrandom.  The result is returned in the counter \randomi.
% Do not change \randomi except, perhaps, to initialize it at some
% random value.  If you do not initialize it, it will be initialized
% using the time and date.  (This is a sparse initialization, giving
% fewer than a million different starting values, but you should use
% other sources of numbers if they are available--just remember that
% most of the numbers available to TeX are not at all random.)
%
% The \nextrandom command is not very useful by itself, unless you
% have exactly 2147483646 things to choose from.  Much more useful
% is the \setrannum command which sets a given counter to a random
% value within a specified range.  There are three parameters:
% \setrannum {<counter>} {<minimum>} {<maximum>}.  For example, to
% simulate a die-roll: \setrannum{\die}{1}{6} \ifcase\die... .
%
% If you need random numbers that are not integers, you will have to
% use dimen registers and \setrandimen.  For example, to set a random
% page width: \setrandimen \hsize{3in}{6.5in}.  The "\pointless" macro
% will remove the "pt" that TeX gives so you can use the dimensions
% as pure `real' numbers.  In that case, specify the range in pt units.
% For example,
%   \setrandimen\answer{2.71828pt}{3.14159pt}
%   The answer is \pointless\answer.
%
% The random number generator is the one by Lewis, Goodman, and Miller
% (1969) and used as "ran0" in "Numerical Recipies" using Schrage's
% method for avoiding overflows.  The multiplier is 16807 (7^5), the
% added constant is 0, and the modulus is 2147483647 (2^{31}-1).  The
% range of integers generated is 1 - 2147483646.  A smaller range would
% reduce the complexity of the macros a bit, but not much--most of the
% code deals with initialization and type-conversion.  On the other hand,
% the large range may be wasted due to the sparse seed initialization.

\newcount\rand@mSeed % set this to have the same random numbers each time
\newcount\randomi % the random number seed (while executing)
\global\randomi\catcode`\@  % scratch variable during definitions
\catcode`\@=11

\def\randomseed#1{\rand@mSeed=#1\relax}

\def\nextrandom{\begingroup
 \ifnum\randomi<\@ne % then initialize with time
    \ifnum\rand@mSeed>0 
      \global\randomi\rand@mSeed
    \else
      \global\randomi\time
    \fi
    \global\multiply\randomi388 \global\advance\randomi\year
    \global\multiply\randomi31 \global\advance\randomi\day
    \global\multiply\randomi97 \global\advance\randomi\month
    \message{Randomizer initialized to \the\randomi.}%
    \nextrandom \nextrandom \nextrandom
 \fi
 \count@ii\randomi
 \divide\count@ii 127773 % modulus = multiplier * 127773 + 2836
 \count@\count@ii
 \multiply\count@ii 127773
 \global\advance\randomi-\count@ii % random mod 127773
 \global\multiply\randomi 16807
 \multiply\count@ 2836
 \global\advance\randomi-\count@
 \ifnum\randomi<\z@ \global\advance\randomi 2147483647\relax\fi
 \endgroup
}

\countdef\count@ii=2 % use only in boxes!
\ifx\@tempcnta\undefined \csname newcount\endcsname \@tempcnta \fi
\ifx\@tempcntb\undefined \csname newcount\endcsname \@tempcntb \fi

\def\setrannum#1#2#3{% count register, minimum, maximum
 \@tempcnta#3\advance\@tempcnta-#2\advance\@tempcnta\@ne
 \@tempcntb 2147483645 %  =  m - 2  =  2^{31} - 3
 \divide\@tempcntb\@tempcnta
 \getr@nval
 \advance\ranval#2\relax
 #1\ranval
}

\def\setrandim#1#2#3{% dimen register, minimum length, maximum length
 \dimen@#2\dimen@ii#3\relax
 \setrannum\ranval\dimen@\dimen@ii
 #1\ranval sp\relax
}

\def\getr@nval{% The values in \@tempcnta and \@tempcntb are parameters
 \nextrandom
 \ranval\randomi \advance\ranval\m@ne \divide\ranval\@tempcntb
 \ifnum\ranval<\@tempcnta\else \expandafter\getr@nval \fi
}

\def\pointless{\expandafter\PoinTless\the}
{\catcode`p=12 \catcode`t=12 
\gdef\PoinTless#1pt{#1}}

\catcode`\@=\randomi
\global\randomi=0
\newcount\ranval
